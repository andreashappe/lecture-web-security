\documentclass[crop,tikz]{standalone}% 'crop' is the default for v1.0, before it was 'preview'
%\usetikzlibrary{...}% tikz package already loaded by 'tikz' option

\usepackage[ngerman]{babel}
\usepackage{tikz-uml}

\begin{document}
\begin{tikzpicture}[scale=\textwidth/15.2cm,samples=200]
	\tikzumlset{fill object = white, fill call = gray!20} 
	\tikzstyle{every node}=[font=\small]

	\begin{umlseqdiag}
		\umlactor[no ddots]{User}
		\umlobject[no ddots]{Browser}
		\umlobject[no ddots]{Target Website}
		\umlobject[no ddots]{Malicious Website}

		\begin{umlcall}[dt=8, op={Normale Verwendung}, return={Session bleibt geöffnet}]{User}{Browser}
			\begin{umlcall}[dt=8, op={Login}, return={Session Daten}]{Browser}{Target Website}
			\end{umlcall}
		\end{umlcall}

		\begin{umlcall}[dt=8, op={Surfen im Internet}, return={}]{User}{Browser}
			\begin{umlcall}[dt=8, op={Waterhole Attack/Social Engineering}, return={Website mit Formular}]{Browser}{Malicious Website}
			\end{umlcall}

			\begin{umlcallself}[dt=8, op={Formular wird automatisch aufgerufen}]{Browser}
			\end{umlcallself}

			\begin{umlcall}[dt=8, op={Form-Operaton wird mit Session aufgerufen}, return={}]{Browser}{Target Website}
				\begin{umlcallself}[dt=8, op={Operation wird als User ausgeführt}]{Target Website}
				\end{umlcallself}
			\end{umlcall}
		\end{umlcall}
	\end{umlseqdiag}
\end{tikzpicture}
\end{document}
