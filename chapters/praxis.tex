\chapter{Introduction}

Während der erste Part primär in die ``Breite'' ging und theoretischer Natur ist, ist dieser Teil nun quasi die Praxis zu dem ersten Teil.

Eventuell mehr Richtung Defense gehen, code-examples wie man Angriffe verhindern kann.

\chapter{XSS Attacks}

\section{Common Patterns}

\section{DOM-based Injections}

\section{Bypassing Filters}

\url{https://null-byte.wonderhowto.com/how-to/advanced-techniques-bypass-defeat-xss-filters-part-1-0190257/}

\url{https://null-byte.wonderhowto.com/how-to/advanced-techniques-bypass-defeat-xss-filters-part-2-0190959/}

\section{Example}

\url{https://cube01.io/blog/Moodle-DOM-Stored-XSS-to-RCE.html}

\chapter{File-Uplodas}

\url{https://pentesterlab.com/exercises/php_include_and_post_exploitation/course}

\chapter{CSRF-Attacks}

Soll ich das überhaupt noch inkludieren? Immerhin sollten demnächst viele Browser das automatisch canceln (SameSite).

\chapter{SQL-Injections}

\section{Union-Based Attacks}

\url{https://pentesterlab.com/exercises/from_sqli_to_shell/course}

\section{Blind SQL-Injection}

\url{https://pentesterlab.com/exercises/from_sqli_to_shell_II/course}

\chapter{XEE-based Attacks}

\chapter{Serilisierungsangriffe}

\chapter{Type Juggling Attacks}

\chapter{JWT-basierte Angriffe}

\chapter{HTTP Request Splitting Attacks}

\url{https://medium.com/@andrewaeva_55205/how-dangerous-is-request-splitting-a-vulnerability-in-golang-or-how-we-found-the-rce-in-portainer-7339ba24c871}
